\documentclass[12pt, a4paper]{article}
\usepackage[left=3cm, right=2cm, top=2cm, bottom=2cm, bindingoffset=0cm]{geometry}
\usepackage{minted}
\usepackage{tikz}
\usepackage[T2A]{fontenc} % Поддержка русских букв
\usepackage[utf8]{inputenc}
\usepackage[english, russian]{babel}
\usepackage{amssymb}
\usepackage{amsmath}
\usepackage{mathcomp}
\usepackage{amsfonts}
\usepackage{xcolor}
\usepackage{hyperref}
\graphicspath{ {./ph/} }
\usepackage{graphicx}
\usepackage{wrapfig}
\definecolor{linkcolor}{HTML}{008080} % цвет ссылок
\definecolor{urlcolor}{HTML}{FF4500} % цвет гиперссылок
\definecolor{bg}{HTML}{F0F8FF}  % цвет подкладки кода
\definecolor{cite}{HTML}{008080} %цвет ссылок на литературу
\hypersetup{pdfstartview=FitH,  linkcolor=linkcolor,urlcolor=urlcolor, citecolor=cite, colorlinks=true}

\DeclareMathOperator{\Mr}{M_{\mathbb{R}}}

\begin{document}
\thispagestyle{empty}
\begin{center}
	\textbf{ПРАВИТЕЛЬСТВО РОССИЙСКОЙ ФЕДЕРАЦИИ}\\
	\vspace{2ex}
	\textbf{Федеральное государственное автономное\\ образовательное учреждение высшего образования}

	\vspace{2ex}

	\textbf{Национальный исследовательский университет \\ <<Высшая школа экономики>>}

	\vspace{8ex}
	\begin{flushright}
	Факультет экономических наук\\
	Образовательная программа <<Экономика>>
	\end{flushright}
\end{center}
\vspace{9ex}

\begin{center}
	{\textbf{КУРСОВАЯ РАБОТА
	}}
	\vspace{1ex}

	На тему <<Применение рекуррентных нейронных сетей для анализа макроэкономических показателей>>
\end{center}
	\vspace{1ex}
\begin{flushright}
	\noindent
	Студентка группы БЭК162\\Сапельникова Мария Васильевна\\
	\vspace{13ex}
	Научный руководитель:\\
	Старший преподаватель Демешев Борис Борисович

\end{flushright}

	\vfill

\begin{center}
		Москва 2018

\end{center}
\newpage
\tableofcontents
\clearpage

\section{Введение}

\subsection{Мотивация}

Произошедшая в начале 2000-х годов революция в машинном обучении, когда ученые из Университета Торонто научились обучать глубокие нейронные сети, послужила началом их активного применения для решения задач из самых разнообразных областей. Теперь перед исследователями открылись новые горизонты: появились алгоритмы, способные быстро, точно и эффективно предсказывать, классифицировать и моделировать.\footnote{\cite{DeepLearning}, глава 1}
С того момента ситуация частично изменилась. В современном мире существует множество задач, решение которых значительно бы упростилось с использованием машинного обучения. Игра на бирже, вычисление кредитоспособности компании, степень влияния внешних изменений на ключевые макроэкономические показатели, предсказание хода развития заболевания или какого-то химического процесса ~--- каждая задача требует своего подхода, высокой точности и минимальной вероятности ошибки. По этой причине требуется непрерывное совершенствование методов машинного обучения, в частности разрешения многих возникающих в процессе разработки проблем.
\subsection{О чем эта работа?}

Основная цель нашей работы заключается в понимании базовых принципов работы нейронных сетей, в частности архитектуры RNN (Recurrent Neural Networks) и применении этих принципов на практике. Задачей будет прогнозирование временных рядов. Рассмотрим различные проблемы, возникающие при построении и обучении, а также попытаемся найти их наиболее оптимальное решение. В качестве источников используются учебники по машинному обучению, статьи, посвященные данной теме и онлайн-курсы для совершенствования (а иногда и приобретения) навыков программирования.
Итогом работы станет инструкция по реализации рекуррентной нейронной сети для анализа временных рядов, дополненная необходимыми теоретическими сведениями.
Предполагается, что читатель знаком с базовыми операциями программирования, курсом математического анализа и заинтересован в рассматриваемой теме.

\section{Теоретическая часть}

\subsection{Общее представление о нейронных сетях}

Прежде чем начать разговор о более узком понятии рекуррентных нейронных сетей, следует подробнее описать механизм обучения и работы нейронных сетей вообще. Как и для большинства изобретений человечества вдохновителем для их создания является природа, а именно человеческий мозг. До сих пор не до конца изученный, он невероятно быстро решает множество вопросов: <<учит>> новые языки, распознает звуки, генерирует речь и многое другое. С большинством задач мозг справляется гораздо лучше программ как по времени, так и по качеству. Нейронные сети, построенные по принципу человеческого мозга, призваны расширить спектр решаемых задач и увеличить производительность. В чем же заключается этот принцип? Мозг строится из специальных клеток~--- нейронов. Каждый нейрон формируется из тела клетки, коротких отростков~--- дендритов и длинных отростков~--- аксонов. Связи между дендритами и аксонами называются синапсами. см Рис.1\footnote{Источник: \cite{neuron}}

\begin{wrapfigure}{r}{0.5\textwidth}
   \centering
   \includegraphics[width=0.4\textwidth]{neuron.jpg}
   \caption{Строение нейрона}
\end{wrapfigure}

Передаваемые по связям импульсы от одного нейрона к другому имеют разный характер, могут активировать следующий или наоборот снизить его активность. При этом нейроны отлично синхронизируются, достигая высокой степени параллелизации. Обобщенная схема работы человеческого мозга выглядит следующим образом: имеются связанные между собой клетки, каждая из которых находится в определенном состоянии, которое может быть изменено состоянием предыдущего. Одной из самых важных характеристик мозга считается пластичность~--- способность быстро <<обучаться>> и подстраиваться под изменения внешней среды. Именно этот принцип используется в нейронных сетях. При этом важно понимать, что основной задачей этих алгоритмов является не максимальное приближение к природному аналогу (это было бы слишком сложно, как минимум потому что мозг не до конца изучен), а несколько абстрактное применение общего принципа работы.

\subsection{Обучение нейронных сетей}

Что значит <<обучить>> нейронную сеть?
Обучение нейронных сетей происходит при помощи тренировочных данных, функции ошибки и алгоритма минимизации.Фактически, предсказания строятся с помощью учета предыдущих значений. Получаются матрицы весов, с которыми эти значения будут учитываться. В конечном итоге, мы получаем вектор предсказаний, те значения, которые построила модель. Чтобы узнать, насколько эти предсказания точны, необходимо вычислить разность между ними и истинными, затем минимизировать полученную функцию. В большинстве задач используются нелинейные функции ошибки, но так или иначе все они зависят от разность между предсказанием и истиной. Алгоритм, в роли которого в современном машинном обучении как правило выступает градиентный спуск, должен минимизировать функцию ошибки, зависящую от двух векторов: вектора предсказания и вектора истинных значений. Вследствие минимизации функции ошибки с помощью алгоритма происходит изменение весов и отклонений (или пороговых элементов) в модели.

Принцип работы этого метода будет описан в разделе <<Проблемы рекуррентных нейронных сетей>>

\subsection{Рекуррентные нейронные сети}

Основная область применения рекуррентных нейронных сетей~--- обработка различных последовательностей, где значение предыдущего элемента связано со следующим и каким-то образом влияет на него. Самым простым примером может стать предсказание последнего слова в предложении. Когда мы читаем предложение, его смысл постепенно становится понятнее с каждым прочитанным словом. То есть мозг не  <<удаляет>>  из памяти предыдущее прочитанное и <<запоминает>> новое, а <<обновляет>> значение всего предложения. Точно таким же образом действуют рекуррентные нейронные сети, учитывая все ранее произошедшие изменения и обновляя таким образом значение результата.
Особенностью устройства рекуррентный нейронных сетей является наличие петель~--- связей нейрона с самим собой.
Так формируются петли. Входящая последовательность разбивается по элементам, каждое внутреннее состояние получает на вход соответствующий элемент и результат предыдущего блока. Как видно на Рис.2 \footnote{Источник: \cite{TensorFlow}, стр. 77} входной вектор $x_t$ обновляет состояние выходного вектора $h_{t-1}$ до $h_t$.
\begin{figure}[ht]
	\centering
	\includegraphics[width=0.7\textwidth]{rnn.jpg}
	\caption{Клетка рекуррентной нейронной сети}
\end{figure}\\

Так, мы можем развернуть любой узел рекуррентной нейронной сети в цепь, состояние которой обновляется с каждым шагом. Наша задача~--- описать этот шаг, а именно как каждое следующее «слово» влияет на смысл. Иными словами, нам нужно научить цепь «думать».\footnote{Источник:\cite{TensorFlow}, стр. 77}
Разберем схему на примере ряда реального ВВП России с 2009 по 2011 год. Имеем входной вектор $x = (38807.2, 46308.5, 55967.2)$. Задача~--- предсказать значение ВВП в 2012 году. Соответственно, $x_{t-1}$, $x_t$ и $x_{t+1}$ будут элементами этого вектора, а требуемое предсказание~--- $h_t$. То, что происходит внутри каждого блока будет описано далее.

\subsection{Математическая интерпретация происходящего}

Опишем каждый блок рекуррентной нейронной сети\footnote{\cite{TensorFlow}, часть 2, глава 6}.
Введем используемые обозначения:\\
$W$~--- матрица весов для перехода между слоями\\
$U$~--- матрица весов для входов\\
$V$~--- матрица весов для выходов\\
$f$~--- функция активации\\
$h$~--- функция для получения ответа
$s_t$~--- состояние сети в момент времени $t$\\
$b$ и $c$~--- отклонения\\
$x_t = (x_1, x_2, x_3)$~--- вектор входящих значений\\
Тогда задача блока в момент времени t выглядит следующим образом:
\begin{equation}\label{formula}
a_t = b + W s_{t-1} + U x_t
\end{equation}
\begin{equation}\label{formula1}
o_t = c + V s_t
\end{equation}
\begin{equation}\label{formula2}
s_t = f(a_t)
\end{equation}
\begin{equation}\label{formula3}
y_t = h(o_t)
\end{equation}
Поступающий на вход элемент $x_t$ умножается на матрицу весов $U$, предыдущее состояние сети $s_{t-1}$~--- на матрицу весов W, добавляется свободный член (смещение, позволяющее «выровнять» значение)\hyperref[formula]{(1)}. Полученное выражение подставляется в функцию активации f, получается состояние сети в момент t~---  $s_t$\hyperref[formula2]{(3)}. Далее это состояние передается на выход с весами V\hyperref[formula1]{(2)} и обрабатывается функцией h, приводящей полученный результат к требуемому задачей виду\hyperref[formula3]{(4)}и в следующий блок с весами W.\\
Отдельное внимание стоит уделить выбору функции активации.\\
Рассмотрим четыре самых часто используемых в машинном обучении: сигмоид ($\sigma$), гиперболический тангенс ($tanh$), ступенчатая функция и ReLU.\\

\begin{itemize}
	\item Сигмоид: $\sigma = \dfrac1{1 + e^{x}}$, Рис.3\\

	Основные свойства: дифференцируемая на всей области значений, при $x \rightarrow -\infty$ стремится к 0, при х $\rightarrow +\infty$ стремится к 1. Существенным недостатком является высокая скорость стремления к крайним точкам, что приводим к быстрому затуханию градиента при обучении. Важность этой проблемы станет более понятной в следующем разделе.

	\begin{figure}[ht]
		\centering
		\includegraphics[width=0.5\textwidth]{sigmoida.jpg}
		\caption{Сигмоид}
	\end{figure}

	\item Гиперболический тангенс: $tanh = \dfrac{e^{x} - e^{-x}}{e^{x} + e^{-x}}$, Рис.4\\

	Основные свойства: во многом похож на сигмоид, везде дифференцируемый, при $x \rightarrow -\infty$ стремится к -1, при х $\rightarrow +\infty$ стремится к 1. Скорость стремления к крайним точкам еще выше, однако среди них нет 0, который обнулял бы общий градиент при обучении. А значит, использование этой функции более разумно. Помимо этого, 0 является центральной точкой, то есть можно начать обучение в этой точек, смещаясь в любую сторону.

	\begin{figure}[ht]
		\centering
		\includegraphics[width=0.5\textwidth]{tanh.jpg}
		\caption{Гиперболический тангенс}
	\end{figure}

	\item Ступенчатая функция (функция Хевисайда), Рис.5 :
$
	\begin{cases}
		0, & x<0 \\
		1, & x>0
	\end{cases}\\
$
	Основные свойства: не определена в нуле, можем доопределить, получим грубое приближение вышеописанных, тогда сможем дифференцировать. Проблема совпадает с сигмоидом~--- нулевое крайнее значение обнуляет общий градиент при обучении.

	\begin{figure}[ht]
		\centering
		\includegraphics[width=0.5\textwidth]{step.jpg}
		\caption{Ступенчатая функция}
	\end{figure}

	\item ReLU, Рис.6:
	$
	\begin{cases}
		0, & x<0 \\
		x, & x\geq0
	\end{cases}\\
	$
	Основные свойства: Простая в дифференцировании, сокращает время на вычисления. Интересно, что именно эта функция наиболее схожа с тем, что происходит в человеческом мозгу \footnote{\cite{DeepLearning}, часть 1, раздел 3.3}. В указанном источнике можно подробно прочитать про нее, мы же не будем вдаваться в глубокие математические подробности.

	\begin{figure}[ht]
		\centering
		\includegraphics[width=0.5\textwidth]{relu.jpg}
		\caption{ReLU}
	\end{figure}


\end{itemize}

В своей работе я буду использовать функцию гиперболического тангенса, так как она наиболее понятная для примеров и не вызывает серьезных проблем при обучении.

\subsection{Проблемы рекуррентных нейронных сетей}

Основным алгоритмом, используемым в обучении нейронных сетей, является метод градиентного спуска. Как известно, градиент функции – вектор, демонстрирующий направление её наискорейшего роста. Соответственно, если взять антиградиент (все элементы вектора-градиента с противоположным знаком), можно узнать направление наискорейшего убывания функции. В задачах обучения, как правило, минимизируется функция ошибки, то есть величина от разности истинного значения и полученного предсказания. Если мы будем перемещаться в направлении антиградиента в каждой точке функции ошибка с каким-то шагом, то рано или поздно достигнем её глобального минимума. Так как шаг является настраиваемым параметром, с ним связаны основные проблемы данного метода. Что будет, если шаг будет выбран слишком маленьким? Слишком большим? Появляется возможность либо «не дойти» до глобального минимума и <<застрять>> в локальном, либо <<перепрыгнуть>> его, продвинувшись слишком далеко.
Природа проблем с обучением рекуррентных нейронных сетей несколько схожа с вышеописанным принципом, но прежде всего возникает вопрос об обучении сетей с такой архитектурой. Не будем вдаваться в подробности метода обратного распространения ошибки, так как он описан в большом количестве статей (источники будут указаны в конце работы). Изучив принцип работы этого метода, необходимо разобраться каким образом передавать значение градиента ошибки от одного нейрона к другому, если они соединены сами с собой? Давайте выпишем выходное значение сети в какой-то момент времени t, используя ранее описанные формулы\hyperref[formula]{(1)}~-- \hyperref[formula3]{(4)}, но без матриц весов и смещений, чтобы не загромождать запись. Пусть $t = 3$ (для простоты).\\
$y_3 = h(x_3, s_2) =
h(x_3, f(x_2, s_1)) =
h(x_3, f(x_2, f(x_1, s_0)))\\
\noindent
s_0$ - начальное состояние сети

Теперь стало понятно, каким образом считать градиент и передавать его значение дальше, а также почему рассматриваемая архитектура называется "рекуррентной".
Так как матрицы весов W и U в каждом блоке одинаковы, а градиент умножается на них с каждым шагом и растет экспоненциально, может возникнуть две основных проблемы: <<взрывающийся>> и <<затухающий>> градиент.\\

\textit{Проблема <<взрывающегося>> градиента»}

Если же матрица весов устроена таким образом, что при умножении на нее норма вектора-градиента возрастает, рано или поздно возникнет проблема <<взрывающегося>> градиента. Тогда с каждым следующим шагом в методе градиентного спуска норма градиента растет экспоненциально, и глобальный минимум функции ошибки не будет достигнут, как и в случае со слишком большим шагом. Решением в данном случае будет простое ограничение значения градиента извне, что просто не даст ему расти выше заданной границы.
\\

\textit{Проблема <<затухающего>> градиента»}

В простых нейронных сетях не учитывается взаимосвязь предыдущих элементов со следующими, поэтому факт того, что градиент затухает, не является проблемой. В рекуррентных особенно важно как можно дольше сохранить <<контекст>>, то есть степень влияния предшествующих элементов ряда. Проблема заключается в том, что если матрица весов устроена таким образом, что при каждом умножении на нее норма вектора-градиента уменьшается экспоненциально, то в какой-то момент $t+i$ она и вовсе будет незначительной величиной. Здесь возникает проблема краткосрочной памяти.
Таким образом, в какой-то момент времени $t$ влияние вектора $x_{t-i}$ будет аналогично незначительным или вовсе не будет учитываться при формировании предсказания. Вполне логично, что чем больше объем контекста, тем точнее будет предсказание последнего элемента, так что наша задача – сохранить влияние как можно большего числа элементов ряда.

Пока не вдаемся в подробности реализации, этому посвящена вся практическая часть работы.
Решение проблемы <<затухающего>> градиента не такое простое. Существует несколько усовершенствованных архитектур (базовым принципом остаются рекуррентные сети), а именно модель долгой краткосрочной памяти LSTM, GRU, SCRN и другие. Рассмотрим каждую из них более подробно в следующем разделе.

\section{Практическая часть. Пробы и ошибки.}
\subsection{Что будем делать?}
Наша задача~--- реализовать обыкновенную рекуррентную нейронную сети на языке Python. Выделим основные этапы:
\begin{itemize}
	\item выбор и установка библиотек
	\item выбор, подготовка и визуализация данных для обучения
	\item инициализация модели RNN
	\item обучение модели
	\item оценка качества работы модели
	\item настройка гиперпараметров
\end{itemize}
Опишем каждый шаг подробнее.
\subsection{Выбор и установка библиотек}

На сегодняшний день существует несколько удобных библиотек для машинного обучения. Все они направлены на упрощение и ускорение решений задач машинного обучения, предоставляя чаще всего используемые готовые функции и методы. Благодаря им можно уместить целый алгоритм в несколько строк кода, а это заметно уменьшает затрачиваемые время и силы.

Мы будем использовать TensorFlow, библиотеку, разработанную в Google. Основными принципом её работы является построение вычислительных графов, с их помощью представляются все операции. Не будем подробно описывать принцип работы, так как существует множество пособий, среди которых \href{https://developers.google.com/machine-learning/crash-course/}{отличный курс от Google Developers}.

Для всех вычислительных операций и работы с TensorFlow необходимы векторы, с этим нам поможет библиотека \href{http://www.numpy.org/}{NumPy} (Numerous Python)~--- мощный инструмент, используемый для решения огромного количества задач.

Для чтения файла с данными и построения таблицы с данными, а также для их первичного анализа используем \href{https://pandas.pydata.org/}{Pandas}.

Чтобы лучше понимать данные, с которыми мы будем работать и отслеживать изменения качества работы модели понадобится визуализация~--- пакет PyPlot из библиотеки \href{https://matplotlib.org/}{Matplotlib}.

Из библиотеки SciKitLearn импортируем функцию для нормализации \href{http://scikit-learn.org/stable/modules/generated/sklearn.preprocessing.MinMaxScaler.html}{MinMaxScaler}, которая приравнивает максимальное значение по выборке к 1, а минимальное~--- к нулю. Будем использовать её для нормировки.

Считаем, что все библиотеки установлены (как это делается можно найти \href{https://pythonworld.ru/osnovy/pip.html}{тут}), а нам остается просто их подключить.

\begin{minted}[bgcolor=bg, linenos=true]{python}
import tensorflow as tf
import numpy as np
import pandas as pd
from matplotlib import pyplot as plt
from sklearn.preprocessing import MinMaxScaler
\end{minted}

\subsection{Выбор, подготовка и визуализация данных}
Так как мы хотим научить нашу сеть прогнозировать временные ряды, нам необходимо обучать её на таких временных рядах, которые бы демонстрировали разное поведение. Для этого хорошо подходит набор данных \href{https://www.m4.unic.ac.cy/the-dataset/}{M4}, будем использовать годовые данные по макроэкономическим показателям.

\begin{minted}[bgcolor=bg, linenos=true]{python}
# Читаем файл.
data = pd.read_csv('Yearly-train.csv')
# Удаляем строки, в которых содержатся значения NaN.
data = data.dropna()
# Обрежем таблицу так, чтобы остались только нужному нам столбцы.
data = np.array(data.iloc[:,1:832])
# Приводим к нужному виду (1, n).
data.reshape(1, -1)
\end{minted}

В результате этих действий мы получаем одну строку размера (1, 831). Изобразим её (реализации этого в работе приводить не будем, в приложении будет весь получившийся код целиком).

\begin{figure}[ht]
	\centering
	\includegraphics[width=150mm]{data.jpg}
	\caption{Визуализация полученных данных}
\end{figure}

Как видно на графике, данные отлично подходят для обучения модели, так как выделяются четыре периода с разным характером поведения (от 0 до 200, от 200 до 400, от 400 до 600 и до конца).

Для обучения нам потребуется выборка с тренировочными и тестовыми данными. Разделим исходную выборку так, что в тренировочной будет 781 элементов, а остальные 50~--- в тестовой.

\begin{minted}[bgcolor=bg, linenos=true]{python}
# Для дальнейшей нормировки сразу приводим к виду (781, 1) и (50, 1).
train_data = data[:,:781].reshape(-1, 1)
test_data = data[:,781:].reshape(-1, 1)
\end{minted}

И тестовую, и тренировочную выборки необходимо пронормировать по выбранному периоду, так как в значениях наблюдается слишком большой разброс: минимальное значение около 1000, а максимальное~--- 11807. Для этого будем использовать описанную выше функцию MinMaxScaler: обработаем ей разбитую на части по 200 элементов выборку для тренировки (не забудем часть, не вошедшую в разбиение) и для тестирования целиком.

\begin{minted}[bgcolor=bg, linenos=true]{python}
scaler = MinMaxScaler()
# Определяем размер периода.
size = 200
for i in range(0, 781, _size):
	scaler.fit(train_data[el:i+size,:])
    train_data[i:i+_size,:] = scaler.transform(train_data[i:i+size,:])
# Обрабатываем оставшуюся часть выборки.
train_data[i:i+size] = scaler.fit_transform(train_data[i:i+size])
# Каждый раз настраиваем функцию по новой выборке.
scaler.fit(test_data)
test_data = scaler.transform(test_data)
# Приводим полученный результат к нужному виду (1, 50, 1).
# Далее будет понятно, почему.
test_data = test_data.reshape(-1, 50, 1)
\end{minted}

Теперь, если мы визуализируем полученный результат, получится следующая картина:
\begin{figure}[ht]
	\noindent\centering
	\includegraphics[width=150mm]{norm_data.jpg}
	\caption{Визуализация нормированных данных}
\end{figure}

Для эффективного обучения модели разделим тренировочные данные на меньшие части, так как если настраивать модель на слишком большом объеме, как такового обучения не произойдет~--- мы предскажем только одно значение. Оценивать качество по такому результату невозможно. Поэтому разделим всю выборку на подвыборки по 50 элементов. Заметим, что столько же значений в  тестовой выборке.

Для удобства в дальнейшем использовании настраиваем форму (None, 50, 1). Значение <<None>>  указывает на то, что мы не знаем, сколько у нас получится строк.
\begin{minted}[bgcolor=bg, linenos=true]{python}
# Количество элементов в одной подвыборке.
num_periods = 50
# Сколько значений "вперед" мы предсказываем.
step = 1
# Приводим всю выборку к одномерному массиву.
train_data = train_data.reshape(-1)
# Подвыборки для входных векторов.
x_data = train_data[:(len(train_data)-(len(train_data) % num_periods))]
x_batches = x_data.reshape(-1, 50, 1)
# Подвыборки для оценки качества (целевые значения).
y_data = train_data[1:(len(train_data)-(len(train_data) % num_periods))+step]
y_batches = y_data.reshape(-1, 50, 1)
\end{minted}
\subsection{Инициализация модели RNN}
Определим необходимые параметры и переменные.
\begin{minted}[bgcolor=bg, linenos=true]{python}
# Количество векторов на вход.
inputs = 1
# Количество клеток в каждом нейроне.
hidden_layers = 100
# Количесвто векторов на выход.
output = 1
x = tf.placeholder(tf.float32, [None, num_periods, inputs])
y = tf.placeholder(tf.float32, [None, num_periods, output])
\end{minted}

Ключевой момент в модели~--- её инициализация. Будем использовать готовые функции TensorFlow, с их документацией можно ознакомиться по ссылкам. Отметим, что по умолчанию \href{https://www.tensorflow.org/api_docs/python/tf/contrib/rnn/BasicRNNCell}{BasicRNNCell} использует в качестве функции активации гиперболический тангенс $tanh$. Специально определим её, так как в работе с гиперпараметрами стоит рассмотреть зависимость качества от выбранной функции. BasicRNNCell создает одну клетку нейрона, \href{https://www.tensorflow.org/api_docs/python/tf/nn/dynamic_rnn}{dynamic rnn} формирует нейросеть целиком.

\begin{minted}[bgcolor=bg, linenos=true]{python}
basic_cell = tf.nn.rnn_cell.BasicRNNCell(num_units = hidden_layers,
activation = tf.nn.tanh)
rnn_output, previous_state = tf.nn.dynamic_rnn(basic_cell, x,
dtype=tf.float32)
\end{minted}

Важным моментом является формирование и настройка формата выходного вектора.

\begin{minted}[bgcolor=bg, linenos=true]{python}
stacked_rnn_output = tf.reshape(rnn_output, [-1, hidden_layers])
stacked_outputs = tf.layers.dense(stacked_rnn_output, output)
# Финальный результат нужной формы (None, 50, 1).
outputs = tf.reshape(stacked_outputs, [-1, num_periods, output])
\end{minted}

\subsection{Обучение модели}
Определим шаг градиентного спуска. В разделе с настройкой гиперпараметров рассмотрим, как его значение влияет на качество предсказаний.
\begin{minted}[bgcolor=bg, linenos=true]{python}
learning_rate = 0.0001
\end{minted}

Для настройки будем пользоваться готовым оптимизатором AdamOptimizer, который работает по принципу обратного распространения ошибки через градиентный спуск. Подробнее о нем можно почитать \href{https://arxiv.org/pdf/1412.6980.pdf}{тут}.

Функцию ошибки определим как среднеквадратичную (MSE, Mean Squared Error) $\sum_{i=1}^n (\hat{y_i} - y_i)^2$, где n - количество предсказаний, $y_i$ - истинные значения, а $\hat{y_i}$ - предсказанные значения. Задача оптимизатора~---  её минимизация.

\begin{minted}[bgcolor=bg, linenos=true]{python}
# Функция ошибки.
# Усредняем квадратичную разность предсказаний и реальных.
# Делим на второй элемент размерности выходного вектора (количество столбцов).
loss = tf.reduce_sum(tf.square(outputs - y))/int(outputs.shape[1])
# Инициализация оптимизатора и его задачи.
optimizer = tf.train.AdamOptimizer(learning_rate = learning_rate)
train_op = optimizer.minimize(loss)
\end{minted}

\subsection{Оценка качества работы модели}

В первую очередь объявим все требующиеся переменные и запустим вычислительный граф.
\begin{minted}[bgcolor=bg, linenos=true]{python}
init = tf.global_variables_initializer()
# Количество итераций.
epochs = 1000
with tf.Session() as sess:
    init.run()
    for ep in range(epochs):
        sess.run(train_op, feed_dict={x: x_batches, y: y_batches})
        if ep % 100 == 0:
            mse = loss.eval(feed_dict={x: x_batches, y: y_batches})
            print (ep, "\tMSE:", mse)
    y_pred = sess.run(outputs, feed_dict={x: test_data})
\end{minted}

Значения ошибок для одного из запусков модели:

\begin{minted}[bgcolor=bg, linenos=true]{python}
0   	MSE: 1.7926495
100 	MSE: 0.1196886
200 	MSE: 0.075741634
300 	MSE: 0.053305633
400 	MSE: 0.044808585
500 	MSE: 0.048330028
600 	MSE: 0.043623384
700 	MSE: 0.03213477
800 	MSE: 0.029414654
900 	MSE: 0.031827863
\end{minted}

Визуализация полученных предсказаний (красным цветом изобразим полученные предсказания, синим~--- реальные значения тестовой выборки)
\begin{figure}[ht]
	\noindent\centering
	\includegraphics[width=150mm]{model.jpg}
	\caption{Визуализация полученных предсказаний, $learning rate = 0.001, f = tanh$}
\end{figure}

В целом, модель работает достаточно качественно, не считая ошибок в последнем периоде. Следует иметь в виду то, что объем нашей выборки сравнительно небольшой, а построенная модель несколько грубая, так что такие отклонения можно считать приемлемыми. Тем не менее, постараемся улучшить картину дальнейшими манипуляциями.

\subsection{Настройка гиперпарметров}

В первую очередь уточним, что такое гиперпараметр. Так называются значения, от варьирования которых изменяется качество работы модели. Подбирается методом проб и ошибок. В нашем случае это:
\begin{itemize}
	\item функция активации (activation)
	\item шаг в методе градиентного спуска (learning rate)
	\item количество клеток в каждом нейроне (hidden layers)
\end{itemize}

Рассмотрим, что будет с качеством при изменении каждого из них.

Заменим функцию активации на $ReLU$.

\begin{figure}[ht]
	\noindent\centering
	\includegraphics[width=150mm]{model_1.jpg}
	\caption{Визуализация полученных предсказаний, learning rate = 0.001, $f = ReLU$}
\end{figure}

Посмотрим на значения MSE:

\begin{minted}[bgcolor=bg, linenos=true]{python}
0   	MSE: 4.592088
100 	MSE: 0.08316497
200 	MSE: 0.068788625
300 	MSE: 0.06168739
400 	MSE: 0.057213753
500 	MSE: 0.053298
600 	MSE: 0.06333121
700 	MSE: 0.058019985
800 	MSE: 0.037902087
900 	MSE: 0.02434863
\end{minted}

Полученные предсказания получились лучше, прогнозируемое поведение сильнее совпадает с реальным, ошибка меньше. Далее попробуем изменить шаг градиентного спуска, используя сначала $ReLU$, а потом $tanh$.

Увеличим, а затем уменьшим шаг градиентного спуска. Обратим внимание на значение MSE~--- если оно будет в какой-то момент колебаться, мы столкнемся с проблемой, когда минимум <<перепрыгивается>>, если значение MSE  <<застывает>> ~---  градиент  <<застрял>> в локальном минимуме:

Значения среднеквадратичной ошибки для learning rate = 0.00001  $f = ReLU$

\begin{minted}[bgcolor=bg, linenos=true]{python}
0   	MSE: 3.3546538
100 	MSE: 2.9543178
200 	MSE: 2.5492237
300 	MSE: 2.118637
400 	MSE: 1.5398437
500 	MSE: 0.6524048
600 	MSE: 0.34827211
700 	MSE: 0.31749552
800 	MSE: 0.29518446
900 	MSE: 0.27794966
\end{minted}

Значения ошибки выросли практически в 10 раз, а график (Рис. 11) демонстрирует, что полученная модель достаточно грубая.

\begin{figure}[ht]
	\noindent\centering
	\includegraphics[width=150mm]{model_relu_00001.jpg}
	\caption{Визуализация полученных предсказаний, learning rate = 0.00001, $f = ReLU$}
\end{figure}

Проверим, настолько ли сильно влияет уменьшение шага при использовании другой функции активации.
Рассмотрим значения среднеквадратичной ошибки для learning rate = 0.00001, $f = tanh$:

\begin{minted}[bgcolor=bg, linenos=true]{python}
0   	MSE: 2.263204
100 	MSE: 0.44168743
200 	MSE: 0.22856939
300 	MSE: 0.19207193
400 	MSE: 0.1704937
500 	MSE: 0.15704815
600 	MSE: 0.14776282
700 	MSE: 0.14076014
800 	MSE: 0.1351383
900 	MSE: 0.13038883
\end{minted}
Аналогично с предыдущим результатом значения ошибки сильно выросли. Стоит отметить, что обучаться наша модель стала медленнее (об этом говорит изменение ошибки на каждом шаге). Вполне возможно, мы <<застряли>> в локальном минимуме. График подтверждает нашу гипотезу о неэффективности уменьшения шага (Рис. 12).

\begin{figure}[ht]
	\noindent\centering
	\includegraphics[width=150mm]{model_tanh_00001.jpg}
	\caption{Визуализация полученных предсказаний, learning rate = 0.00001, $f = tanh$}
\end{figure}

Пойдем по другому пути и увеличим значение learning rate до 0.01.
Проверим эффективность при использовании функции активации ReLU:

Значения среднеквадратичной ошибки для Learning rate = 0.01 и $f = ReLU$

\begin{minted}[bgcolor=bg, linenos=true]{python}
0   	MSE: 36.208885
100 	MSE: 0.083905205
200 	MSE: 0.06306617
300 	MSE: 0.054982495
400 	MSE: 0.060820855
500 	MSE: 0.06665602
600 	MSE: 0.050509673
700 	MSE: 0.0509433
800 	MSE: 0.044805717
900 	MSE: 0.044100452
\end{minted}

Ошибка стала меньше, чем в прошлом эксперименте, однако немного выше, чем в самом первого варианта с learning rate = 0.001. Однако на графике (Рис.13) явно заметно улучшение: предсказанные значения чаще совпадают с реальными.

\begin{figure}[ht]
	\noindent\centering
	\includegraphics[width=150mm]{model_relu_01.jpg}
	\caption{Визуализация полученных предсказаний, learning rate = 0.01, $f = ReLU$}
\end{figure}

Для полноты проводимого опыта рассмотрим результат при использовании функции активации tanh:

Значения среднеквадратичной ошибки для Learning rate = 0.01 и $f = tanh$

\begin{minted}[bgcolor=bg, linenos=true]{python}
0   	MSE: 0.27224845
100 	MSE: 0.07114297
200 	MSE: 0.06425515
300 	MSE: 0.061823256
400 	MSE: 0.05978895
500 	MSE: 0.064860746
600 	MSE: 0.057521738
700 	MSE: 0.060832363
800 	MSE: 0.05663346
900 	MSE: 0.058352303
\end{minted}

Ошибка не дает явного ответа на вопрос об улучшении качества, но совсем немного хуже предыдущей. А график (Рис. 14) демонстрирует немного большую неточность.

\begin{figure}[ht]
	\noindent\centering
	\includegraphics[width=150mm]{model_tanh_01.jpg}
	\caption{Визуализация полученных предсказаний, learning rate = 0.01, $f = tanh$}
\end{figure}

Остановим наш выбор на функции активации $ReLU$ и шаге градиентного спуска 0.01. Это не означает, что такие параметры стоит выбирать всегда~---  все зависит от данных и модели.

\section{Вывод}

Как итог проделанной работы еще раз выделим основные результаты. Мы ознакомились с теоретическими понятиями, связанными с рекурреннтынми нейронными сетями, разобрались с основой каждого нейрона (функции активации), а также построили собственную простую модель, предсказывающую временной ряд. Дальнейшей деятельностью по этим направлениям может быть разнообразное улучшение полученной модели. Применение LSTM, GRU, анализ текстовых последовательностей и многое другое может стать предметом изучения и позволит улучшить полученный результат.

Весь код написанной сети и файл с данными можно найти \href{https://github.com/maschasap/rnn.git}{тут}.
\newpage

\addcontentsline{toc}{section}{Список литературы}
\bibliographystyle{utf8gost705u}  % стилевой файл для оформления по ГОСТу
\bibliography{biblio}     % имя библиографической базы (bib-файла)
\end{document}
